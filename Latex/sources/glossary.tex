\newglossary[llg]{glossary}{llo}{lls}{Glossar}
\newglossaryentry{Raspberry Pi}
{
	name=Raspberry Pi,
	description={Einplatinen-Computer, welcher häufig für die Prototypentwicklung eingesetzt wird},
    type=glossary
}
\newglossaryentry{Tensorflow Object Detection API}
{
	name=Tensorflow Object Detection API,
	description={Tensorflow basiertes Framework für die Entwicklung von Objekterkennungsmodellen},
    type=glossary
}
\newglossaryentry{Tensorflow Lite}
{
	name=Tensorflow Lite,
	description={Tensorflow basiertes Framework für die Anwendung von Tensorflow auf mobilen und eingebetteten Systemen},
    type=glossary
}
\newglossaryentry{Tensorflow Lite for Microcontrollers}
{
	name=Tensorflow Lite for Microcontrollers,
	description={Tensorflow Lite basiertes Framework für die Ausführung von Modellen auf Mikrocontrollern},
    type=glossary
}
\newglossaryentry{Tensorflow}
{
	name=Tensorflow,
	description={Bibliothek für maschinelles Lernen und die Entwicklung von Modellen},
    type=glossary
}
\newglossaryentry{Feature Map}
{
	name=Feature Map,
	description={meist zweidimensionale Repräsentation von Merkmalen eines Datums, zum Beispiel eines Bildes},
    type=glossary
}
\newglossaryentry{Array}
{
	name=Array,
	description={Datenstruktur der Programmiersprache C++, welche mehrere gleichartig Objekte enthält},
    type=glossary
}
\newglossaryentry{Map}
{
	name=Map,
	description={Datenstruktur der Programmiersprache C++, welche mehrere gleichartig Objekte enthält, auf die jeweils durch einen Schlüssel zugegriffen wird},
    type=glossary
}
\newglossaryentry{Supervised Learning}
{
	name=Supervised Learning,
	description={überwachtes Lernen},
    type=glossary
}
\newglossaryentry{Unsupervised Learning}
{
	name=Unsupervised Learning,
	description={unüberwachtes Lernen},
    type=glossary
}
\newglossaryentry{Char-Array}
{
	name=Char-Array,
	description={Array aus Chars, ein Char repräsentiert in C++ ein Zeichen und ist ein 8-Bit Datentyp},
    type=glossary
}
\newglossaryentry{SSD}
{
	name=SSD,
	description={Single-Shot Detector, Architektur eines Objekterkennungsmodells, ohne separaten Algorithmus zur bestimmung von möglichen Objektregionen},
    type=glossary
}
\newglossaryentry{MobileNet}
{
	name=MobileNet,
	description={Architektur für Neuronale Netze},
    type=glossary
}
\newglossaryentry{CenterNet}
{
	name=CenterNet,
	description={Architektur für Neuronale Netze},
    type=glossary
}
\newglossaryentry{reinterpretCast}
{
	name=reinterpret\_cast,
	description={wird in der Programmiersprache C++ dazu genutzt den Compiler an zu weisen Daten in einem Speicherbereich als einen bestimmten Typ zu behandeln},
    type=glossary
}
\newglossaryentry{public}
{
	name=public,
	description={Schlüsselwort der Programmiersprache C++, das angibt, dass eine Methode auch von Objekten anderer Klassen aufgerufen werden kann},
    type=glossary
}
\newglossaryentry{protected}
{
	name=protected,
	description={Schlüsselwort der Programmiersprache C++, das angibt, dass eine Methode nur von Objekten von Klassen aufgerufen werden kann, welche von der Klasse der Methode erben},
    type=glossary
}
\newglossaryentry{Smart-Pointer}
{
	name=Smart-Pointer,
	description={Zeigerdatentyp der Programmiersprache C++, welcher die Speicherverwaltung in C++ vereinfacht},
    type=glossary
}