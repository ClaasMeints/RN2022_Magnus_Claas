\chapter{Ausblick}\label{ch:Ausblick}
    Wie im Kapitel~\nameref{sec:Bewertung} beschrieben, ist die Verwendung neuer Hardware der Lösungsansatz mit dem erwartbar geringsten Aufwand. Da das ESP32-S3-EYE Modul aktuell noch nicht vertrieben wird, muss allerdings mit der Erprobung des Ausführung des Modells noch auf den Verkaufsstart gewartet werden. Es ist zu erwarten, dass der Fehler sich nicht verändert. Auch die Laufzeit sollte sich maximal geringfügig ändern, da das Modul im Vergleich zum ESP32-Cam einen etwas moderneren Prozessor verwendet. Der Preis des ESP32-S3-EYE Moduls wird wahrscheinlich höher als der Preis des ESP32-Cam Moduls ausfallen.\\ Betrachtet man die Resultate im Kapitel~\nameref{sec:Bewertung} wird deutlich, dass eine Portierung von SiGI entweder durch eine Anpassung der Modell-Architektur oder durch eine Anpassung der Hardware möglich ist. Sowohl auf Basis des ESP32-Cam Moduls als auch auf Basis des ESP32-S3-EYE Moduls ist die Entwicklung eines Gerätes zur Erfassung analoger Zählerstände grundsätzlich machbar. Sobald das ESP32-S3-EYE Modul verfügbar und der Preis bekannt ist, kann von Seiten des Projektteams entschieden werden, ob die Entwicklung eines zweiten SiGI Modells für das günstigere Modul wirtschaftlich ist.\\ Die Kosten des gesamten Gerätes können auf Basis beider Module in jedem Fall deutlich gesenkt werden. Dabei ist zu entscheiden, wie relevant der preisliche Unterschied zwischen beiden Modulen ist und ob sich das günstigere Modul tatsächlich signifikant auf den Rollout auswirkt. Ziel ist es, das Gerät möglichst breit und möglichst häufig einzusetzten, um eine optimale Datenbasis für digitale Geschäftsmodelle bereitzustellen.