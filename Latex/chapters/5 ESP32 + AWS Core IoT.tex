\chapter{ESP32 + \acrshort{aws} Core \acrshort{iot}}\label{ch:5}
\section{\acrshort{aws} Core \acrshort{iot}}\label{sec:5.1}
Wie bereits erwähnt kann für die Anbindung eines \acrshort{iot}-Geräts an die \acrshort{aws}-\Gls{Cloud} der \acrshort{aws} Core \acrshort{iot} Service genutzt werden. Dieser stellt die Basis aller \acrshort{iot} bezogenen \acrshort{aws} Dienste. Damit ein Gerät angebunden werden kann, muss dieses zunächst als \acrshort{iot} Core Gerät in der \acrshort{aws} Console angelegt werden. Bei diesem Prozess werden Zertifikate erstellt, mit denen sich ein Gerät bei \acrshort{aws} authentifizieren kann. Diese können heruntergeladen und in die Gerätesoftware eingebettet werden\cite*[]{AWSIoT}.\\ Um Gerätesoftware für die Anbindung an \acrshort{aws} Core \acrshort{iot} zu entwickeln, bietet \acrshort{aws} unterschiedliche Frameworks und Software Development Kits an. Diese beinhalten vorgegeben Schritte um die Zertifikate einzubinden. Auch die Kommunikation über MQTT mit der \acrshort{aws}-\Gls{Cloud} kann über vorgefertigte Bibliotheken geschehen\cite*[]{AWSSDK}.\\ Bei der Erstellung des Gerätes in der \acrshort{aws} Console wird bereits für jedes Gerät eine API am \acrshort{aws} Core IoT Endpunkt geschaffen. Darüber hinaus bietet der Endpunkt unter \glqq{}/mqtt\grqq\ bereits einen MQTT-Server auf dem \Gls{Topics} erstellt werden können. Mit dem in der Console verfügbaren MQTT-Test-Client kann die Verbindung zu den einzelnen Geräten getestet werden.
\section{Demo-Projekt Quizz-Buzzer}\label{sec:5.2}
Zur demonstration der Funktionen von \acrshort{aws} Core \acrshort{iot} wird ein Quizz-Buzzer auf Basis eines ESP32 als \acrshort{iot}-Gerät entwickelt. Konkret wird das Entwicklungsboard \glqq{}ESP32-S devkitC V4\grqq\ mit dem ESP-WROOM-32U verwendet. Das Modul verfügt über eine WLAN Schittstelle und zwei Prozessorkerne, sowie eine Vielzahl digitaler Ein- und Ausgänge. Wie in Abbildung~\ref*{fig:5.1} dargestellt werden zwei LEDs über zwei digitale Ausgänge beschaltet und vier Taster an digitale Eingänge geschaltet. 
\fig{5.1}{Buzzer auf Basis eines ESP32 (Eigendarstellung)}\FloatBarrier