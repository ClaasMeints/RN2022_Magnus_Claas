\chapter{\acrfull{iot}}\label{ch:3}
\section{Was ist \acrshort{iot}}\label{sec:3.1}
Unter \acrshort{iot} wird die Vernetzung vieler physicher Geräte mittels dem Internet verstanden. Der Begriff ist bereits seit 1999 in Gebrauch\cite*[]{IARJSET} und wird für eine vielzahl unterschiedlicher technischer Lösungen und Protokolle verwendet. Die verschiedenen technischen Lösungen eint dabei die Verwendung des \acrlongpl{ip} (\acrshort{ip}) innerhalb des \Gls{Protokoll-Stack}s\cite*[]{Rayes2019}.\\ Der erste Schritt in Richtung \acrshort{iot} ist dabei für gewöhnlich die Ausstattung von Geräten mit einer vielzahl von Sensoren. So können Geräte den Zustand ihrer Umgebung erfassen und diese über die Netzwerkverbindung an ein \Gls{Backend} übertragen. Dadurch wird es möglich physische Zustände lokal, aber potenziell auch von überall auf der Welt aus zu überwachen. Das Backend kann also zum einen einer Privatperson ermöglichen die eigenen Geräte zu überwachen, zum anderen können die Geräte auch als Datenquellen zum Beispiel für Unternehmen dienen, welche dann die Informationen vieler Geräte in einem \Gls{Cloud}-\Gls{Backend} bündeln\cite*[]{inproceedings}.\\ Für viele technische Anwendungen ist neben Sensordaten auch die Möglichkeit der Steuerung von Aktoren notwendig. Hierbei ist sowohl die Vernetzung mit anderen \acrshort{iot}-Geräten als auch die Vernetzung mit dem \Gls{Backend} wichtig. Über die Kopplung mit einer \acrshort{iot}-Sensoreinheit kann so zum Beispiel die Regelung des Aktors vorgenommen werden. Wird der Aktor an ein Backend angeschlossen, ist darüber eine automatische Fernsteuerung möglich\cite*[]{Rayes2022}.\\ Neben Sensoren und Aktoren sind Einheiten für die Verarbeitung von Daten notwendig. Dabei kann die Datenverarbeitung sowohl im \Gls{Backend} geschehen, als auch durch ein \acrshort{iot}-Gerät. Die Verarbeitungslogik kann sowohl in ein bestehendes \acrshort{iot}-Gerät integriert werden, als auch als separate Einheit ausgeführt sein. Durch die so geschaffene Möglichkeit der lokalen Verarbeitung von Daten wird das Gesamtsystem resilienter gegen den Ausfall von Netzwerkverbindungen. Auch die Latenz wird durch die direktere Verbindung im lokalen Netzwerk verringert\cite*[]{Rayes2022}.
\section{Anwendungsgebiete}\label{sec:3.2}
Durch die bereits beschriebene Vielfältigkeit der technischen Lösungen, die unter dem Begriff \acrshort{iot} zusammengefasst werden ergibt sich auch eine Vielzahl an Anwendungsgebieten und Einsatzmöglichkeiten.\\ Einer der präsentesten Bereiche in dem \acrshort{iot}-Geräte eingesetzt werden ist die Hausautomatisierung. Hier eingesetzte \acrshort{iot}-Geräte können zum einen dem Komfort der Benutzer*innen erhöhen, in dem Arbeiten im Haushalt automatisiert oder erleichtert werden. Darüber hinaus existieren bereits viele Lösungen, welche primär der Sicherheit gegen Einbruch etc. dienen. Eine weitere wichtige Aufgabe von \acrshort{iot}-Geräten im Haus ist die energetische Optimierung. So werden \glqq{}intelligente\grqq\ Heizsysteme mit Temperaturfühlern und Aktoren an den Rolläden gekoppelt, um die gewünschte Temperatur im Innenraum möglichst kostengünstig und Emissionsarm zu erreichen. Auch die Anpassung des elektrischen Energieverbrauchs an die Verfügbarkeit und den Preis von elektrischer Energie kann mittels \acrshort{iot} erfolgen\cite*[]{Zaheeruddin2020}.\\ Auch für die Sammlung medizinisch relevanter Daten werden vermehrt \acrshort{iot}-Geräte eingesetzt. Ein Beispiel kann die Messung von Blutruck oder die Durchfühung eines Elektrokardiogramms durch ein Armband sein. Da diese permanent getragen werden, können Gesundheitliche Probleme auch ohne Verdachtsfall erkannt werden. Eine rechtzeitige Behandlung wird so wahrscheinlicher\cite*[]{Zaheeruddin2020}.\\ In der Landwirtschaft können \acrshort{iot}-Geräte dazu verwendet werden zum Beispiel das Pflanzenwachstum zu überwachen. Durch die Erfassung der Daten kann optimal bewässert und gedüngt werden. Auch vollautomatische Systeme können mittels Steuerungstechnik realisiert werden. \acrshort{iot} bietet dabei die Möglichkeit Sensordaten zu erfassen und im \Gls{Cloud}-\Gls{Backend} mit Wetterdaten zu verknüpfen um langfristig die Bewässerung planen und Optimieren zu können\cite*[]{Zaheeruddin2020}.\\ Der Vermutlich wichtigste Einsatzbereich von \acrshort{iot} ist die Industrie. Der Einsatz von \acrshort{iot}-Geräten in der Industrie wird in Deutschland mit dem Begriff \glqq{}Industrie 4.0\grqq\ beschrieben. Mögliche aufgaben von \acrshort{iot}-Geräten sind die Bereitstellung von Daten zur Bestimmung wartungsintervallen oder aber auch die Automatisierung ganzer Produktionsstraßen. Viele Unternehmen streben dabei eine sogenannte \glqq{}Smart Factory\grqq\ an, also eine Produktionsstraße, welche unterschiedliche Produkte fertigen kann und die Produktionsdaten von einem oder mehreren in dem Produkt integrierten \acrshort{iot}-Geräten erhält. Alle an dem Fertigungsprozess beteiligten Geräte sind dabei untereinander und mit dem Produkt vernetzt. Im idealfall ist für ein Angepasstes Produkt nur eine Änderung der im Produkt gespeicherten Produktionsdaten notwendig, ohne dass die am Fertigungsprozess beteiligten Geräte neu Programmiert oder anderweitig angepasst werden müssen\cite*[]{Zaheeruddin2020}\cite*[]{10.1007/978-3-030-39875-0_20}.\\ % TODO: China?
\section{Technologische Umsetzung von \acrshort{iot}}\label{sec:3.3}
Da eine Ethernet-Verbindung für \acrshort{iot}-Geräte oft nicht verfügbar oder für den Anwendungsfall ungeeignet ist, kommunizieren ein Großteil der \acrshort{iot}-Geräte Drahtlos. So können auf den Layern Eins und Zwei zum Beispiel die Funkstandards IEEE 802.15.4 oder LPWAN bzw. LoRaWAN eingesetzt werden.