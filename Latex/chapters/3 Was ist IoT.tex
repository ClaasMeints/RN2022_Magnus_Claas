\chapter{\acrfull{iot}}\label{ch:3}
\section{Was ist \acrshort{iot}}\label{sec:3.1}
Unter \acrshort{iot} wird die Vernetzung vieler physicher Geräte mittels dem Internet verstanden. Der Begriff ist bereits seit 1999 in Gebrauch\cite*[]{IARJSET} und wird für eine vielzahl unterschiedlicher technischer Lösungen und Protokolle verwendet. Die verschiedenen technischen Lösungen eint dabei die Verwendung des \acrlongpl{ip} (\acrshort{ip}) innerhalb des \Gls{Protokoll-Stack}s\cite*[]{Rayes2019}.\\ Der erste Schritt in Richtung \acrshort{iot} ist dabei für gewöhnlich die Ausstattung von Geräten mit einer vielzahl von Sensoren. So können Geräte den Zustand ihrer Umgebung erfassen und diese über die Netzwerkverbindung an ein \Gls{Backend} übertragen. Dadurch wird es möglich physische Zustände lokal, aber potenziell auch von überall auf der Welt aus zu überwachen. Das Backend kann also zum einen einer Privatperson ermöglichen die eigenen Geräte zu überwachen, zum anderen können die Geräte auch als Datenquellen zum Beispiel für Unternehmen dienen, welche dann die Informationen vieler Geräte in einem \Gls{Cloud}-\Gls{Backend} bündeln\cite*[]{inproceedings}.\\ Für viele technische Anwendungen ist neben Sensordaten auch die Möglichkeit der Steuerung von Aktoren notwendig. Hierbei ist sowohl die Vernetzung mit anderen \acrshort{iot}-Geräten als auch die Vernetzung mit dem \Gls{Backend} wichtig. Über die Kopplung mit einer \acrshort{iot}-Sensoreinheit kann so zum Beispiel die Regelung des Aktors vorgenommen werden. Wird der Aktor an ein Backend angeschlossen, ist darüber eine automatische Fernsteuerung möglich.\\ Neben Sensoren und Aktoren sind Einheiten für die Verarbeitung von Daten notwendig. Dabei kann die Datenverarbeitung sowohl im \Gls{Backend} geschehen, als auch durch ein \acrshort{iot}-Gerät. Die Verarbeitungslogik kann sowohl in ein bestehendes \acrshort{iot}-Gerät integriert werden, als auch als separate Einheit ausgeführt sein. Durch die so geschaffene Möglichkeit der lokalen Verarbeitung von Daten wird das Gesamtsystem resilienter gegen den Ausfall von Netzwerkverbindungen. Auch die Latenz wird durch die direktere Verbindung im lokalen Netzwerk verringert.
\section{Anwendungsgebiete}\label{sec:3.2}
\section{Technologische Umsetzung von \acrshort{iot}}\label{sec:3.3}