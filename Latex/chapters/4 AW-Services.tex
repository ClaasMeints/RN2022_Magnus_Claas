\chapter{Amazon Web Services}\label{ch:4}
\acrfull{aws}

Das Unternehmen Amazon bietet selbst mehr als Erfolgreich Cloud Computing Dienste unter dem Namen AmazoncWe Services (AWS) an. Anfänglich wurden nicht genutzte Rechenkapazitäten verkauft, jedoch hat sich das Anbieten von IT-Ressourcen in der Cloud als extrem profitabel erwiesen und mittlerweile erzielt das Cloud Geschäft des Unternehmens mehr Gewinn als der bekannte Online Handel. Bei AWS handelt es sich aktuell um den Cloud Anbieter mit dem weltweit größten Marktanteil und derzeit werden den Benutzern mehr als 200 verschiedene Dienste angeboten. Somit stehen den Benutzern alle erdenklichen Services zur Verfügung. [https://de.statista.com/infografik/20802/weltweiter-marktanteil-von-cloud-infrastruktur-dienstleistern/
] 
Im Bereich IaaS bietet der Anbieter beispielsweise Services wie EC2 und S3 an. Mit EC2 wird dem Benutzer Rechenleistung in Form von Instanzen auf dem Server bereitgestellt und dieser hat dabei die vollständige Kontrolle über die virtuellen Ressourcen. Der Benutzer bezahlt hier lediglich die in Anspruch genommene Rechenleistung. Amamzons simple Storage Service S3 bietet einen Objektspeicher mit Web-Schnittstellen. Es ist möglich große Datenmengen schnell in den Datenspeicher hinzuzufügen oder sie abzurufen. Neben S3 beitet Amazon noch weitere Speicherdienste wie z.B. Backup, welche weniger schnelle Zugriffe ermöglichen aber dadurch auch erhebich günstiger pro genutzte Speicherkapazität sind. 

Auch oberhalb der genannten Ebenen bietet AWS Services wie Lambda an. Dieser ist oberhalb von SaaS in der Ebene Function as a Service einzuordnen, denn der Benutzer besitzt weniger Verantwortung als bei SaaS. Mithilfe von Lambda ist es möglich Code einfach Serverseitig durch defiierte trigger ausführen zu lassen. Der Kunde zahlt hier je nachdem wie oft der Code ausgeführt wird.

In Bezug auf das Thema IoT stellt Amazon eine Flotte an Services bereit. FreeRTOS und GreenGrass stellen hier Betriebssysteme dar, welche auf den IoT Geräten laufen und bereits vordefinierte Schnittstellen zur Kommunikation mit der Cloud besitzen. FreeRTOS dient hier als Echtzeitbetriebssystem für die einzelnen IoT Geräte und GreenGrass eher als ein Verwaltungsbetriebssystem z.B. zum Sammeln und Synchronisieren der erfassten Daten auf einem übergeordneten Gerät.

Um eine Steuerung und Überwachung in der Cloud zu ermöglichen dienen Dienste wie IoT Core, Fleetwise und Device Manegement. Durch diese lassen sich Flotten an IoT Geräten verwalten, Überwachen und auch Steuern. Zudem ermöglichen diese ein Sammeln der Daten und ermöglichen ein Updaten der Gerätesoftware von zentraler Stelle.

Zur weiteren Verarbeitung der Daten gibt es weitere Services wie IoT Sitwise, Analytics und Events. Beispielsweise lassen sich Daten für das weitere Verarbeiten mithilfe von MAchine learning vorbereiten durch Filtern, Transformieren und Anreichern. Aus den gesammelten Daten lassen sich Aktionen erstellen, die dann wieder von den IoT Gerätenausgeführt werden. Ein Anwendungsfall hierfür ist eine Maschine in einer Produktionshalle. Durch das Auswerten der gesammelten Daten der gesamten Sensorik, lassen sich Rückschlüsse auf den Zustand der Mdchine schließen. Somit ist es möglich vollstöndig automatisiert in die Produktuion einzugreifen sobald eine Unregelmäßigkeit im Betrieb festgestellt wird.