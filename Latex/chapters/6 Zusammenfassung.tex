\chapter{Zusammenfassung}\label{ch:Zusammenfassung}
	Ziel dieser Arbeit war die Durchführung einer Machbarkeitsanalyse für die Portierung eines auf einem Objekterkennungsmodell basierenden Algorithmus zum Auslesen analoger Zähler auf einen Mikrocontroller. Entscheidend war dabei, ob es möglich ist, das Objekterkennungsmodell trotz des auf Mikrocontrollern nur begrenzt zur Verfügung stehenden Arbeitsspeichers zu laden und auszuführen.\\ Für die Prüfung der Machbarkeit wurden verschiedene Konzepte entwickelt, erprobt und bewertet. Zum einen wurde betrachtet, wie weit sich das bestehende Objekterkennungsmodell komprimieren lässt, zum anderen wurde geprüft, welche Modell-Architekturen sich noch weiter komprimieren lassen. Auch die Auswahl von Hardware wurde in Bezug auf die Möglichkeiten und Grenzen für die Ausführung eines Objekterkennungsmodells untersucht. Darüber hinaus wurde ein Verfahren zur Fragmentierung von Modellen entworfen. Dieses dient dazu, die Anforderungen, die ein Modell an die Größe des Arbeitsspeichers der Hardware hat, bei konstanter Modellgröße zu reduzieren.\\ Alle Konzepte sind, teilweise zwar mit starken Einschränkungen, grundsätzlich dazu geeignet, die Lücke zwischen dem benötigten Arbeitsspeicher und dem verfügbaren Arbeitsspeicher zu minimieren. Insbesondere mit der zukünftigen Verwendung von momentan noch nicht verfügbarer Hardware in Kombination mit einer Quantisierung des Modells ist eine Lösung mit sehr geringem Arbeitsaufwand sowie sehr geringen Kosten in Aussicht. Sowohl mit diesem Lösungsansatz als auch mit einer veränderten Modellarchitektur lässt sich die Lücke zwischen dem benötigten und dem verfügbaren Arbeitsspeicher schließen.\\ Damit ist die Portierung des Algorithmus machbar. Die Ergebnisse dieser Arbeit können von dem Projektteam als Entscheidungsgrundlage für oder gegen eine weitere Verfolgung eines der Konzepte dienen. Die spezifischen Vorteile bzw. Nachteile der jeweiligen Lösungsansätze sind in einem Vergleich gegenübergestellt.
	\glsaddall