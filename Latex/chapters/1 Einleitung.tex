\chapter{Einleitung}\label{ch:Einleitung}
    Mit Fortschreiten der Digitalisierung rücken datengenstützte, digitale Geschäftsmodelle immer weiter in den Fokus unternehmerischen Handelns. In den vergangenen Jahren sind in fast allen Wirtschaftsbereichen Daten als wichtige und wertvolle Ressource erkannt worden. Daten bilden häufig die Grundlage für neue Lösungsstrategien für bestehende Probleme und damit auch die Basis für neue Geschäftsmodelle.\\ Auch in der Energiewirtschaft gibt es ein Potenzial, viele Informationen aus Daten zu gewinnen und mit ihnen nicht nur energiewirtschaftliche Probleme anzugehen. Besonders interessante Ressourcen stellen die Daten der Energiezähler sowohl von gewerblichen Verbrauchern, aber auch von privaten Haushalten dar. So gibt es innerhalb des EWE Konzerns seit langem nicht nur den Wunsch, die Kosten für die Ablesung von Energiezählern durch Automatisierung zu senken, sondern auch verschiedene Ideen zur Entwicklung neuer Geschäftsmodelle auf Basis der Daten. Die Anwendungsmöglichkeiten hoch aufgelöster Verbrauchsdaten aller Sparten reichen von neuen Tarifmodellen bis hin zur Erkennung von An- bzw. Abwesenheit in einem Haushalt.\\ Das SiGI (Simple Generic Interpreter) Projekt, dem diese Arbeit angeschlossen ist, verfolgt das Ziel, hoch aufgelöste Verbrauchsdaten aller Sparten flächendeckend verfügbar zu machen. Konkret geht es um die Entwicklung eines Gerätes, welches die Messdaten analoger Zähler digital erfasst und diese über ein Backend bereitstellen kann. Der Fokus liegt dabei auf den Sparten Gas und Wasser, während in der Strom-Sparte bereits andere Lösungen etabliert werden.\\ Um eine generische Erfassung von Messdaten aller analogen Zählertypen zu gewährleisten, basiert das im Rahmen des SiGI Projektes entwickelte Gerät auf der Erfassung der von den Messgeräten angezeigten Zählerstände mittels KI-Bilderkennung. In einem Foto von einem Anzeigeinstrument des Zählers müssen dazu Ziffern erkannt werden. Diese Aufgabe wird durch ein Neuronales Netz (Modell) bewältigt.\\ Da für einen flächendeckenden Einsatz des Gerätes möglichst geringe Herstellungskosten erforderlich sind, sollte die im Gerät verwendete Hardware möglichst günstig sein. Ziel dieser Arbeit ist es, zu prüfen, ob die Entwicklung des Gerätes auf Basis eines Mikrocontrollers möglich ist. Ließe sich das Gerät auf Basis eines Mikrocontrollers realisieren, würde dies zu einer signifikanten Senkung der Kosten führen.\\ Ein besonderer Fokus in dieser Arbeit liegt auf der Prüfung, ob sich das im Rahmen von SiGI entwickelte Modell auf einem Mikrocontroller ausführen lässt. Die Detektion von Objekten in einem Bild, zu der auch die Detektion von Ziffern zählt, konnte bisher nicht auf einem Mikrocontroller realisiert werden.\\ Diese Arbeit ist in 6 Kapitel unterteilt. Im Kapitel~\nameref{ch:TGrundlagen und Stand der Technik} werden zunächst alle für die Arbeit relevanten Informationen gesammelt. Das Kapitel~\nameref{ch:Konzeption} beschreibt das Vorgehen bei der Problemlösung und die Konzepte, die in diesem Zuge entwickelt werden. Im Kapitel~\nameref{ch:Ergebnisse} werden die Ergebnisse der Arbeit betrachtet und bewertet und die erarbeiteten Konzepte verglichen. Das Kapitel~\nameref{ch:Ausblick} betrachtet die Folgen für das Projekt SiGI und die \nameref{ch:Zusammenfassung} fasst diese Arbeit zusammen. 
    \pagenumbering{arabic}