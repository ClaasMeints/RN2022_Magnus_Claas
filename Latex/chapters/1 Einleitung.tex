\chapter{Einleitung}\label{ch:1}
\pagenumbering{arabic}

Der Begriff des Cloud Computings genießt aktuell einen hohen Stellenwert in der Informationstechnologie und verändert die IT Landschaft maßgeblich. Seit mehr als 10 Jahren schreitet das Anbieten von IT-Dienstleistungen in der Wolke kontinuierlich voran und die absolute Mehrheit an IT-Unternehmen weltweit nutzt Cloud Computing angebote oder denkt über diese nach\cite*[]{Cloud}. Insbesondere für kleine und mittelständische Unternehmen beietet das Auslagern von Rechenleistung und weiteren Diensten interessante Möglichkeiten. Denn die Anschaffung und Wartung eines eigenen Rechenzentrums stellt kleiner Unternehmen vor eine Herausforderung\cite*[]{DAAS}. Doch nicht für Unternehmen bietet die Cloud neue Möglichkeiten, sondern auch Privatleute profitieren von den Vorteilen der Cloud. Die Meisten Privatpersonen verbinden Cloud mit der Sicherung von ihren Daten auf einem Server, jedoch nutzen diese, ohne es zu wissen, bereits tagtäglich Cloud Dienste.%TODO: Quelle

Neben der Cloud existiert aktuell ein weiterer Trend der ebenfalls eine hohe Aufmerksamkeit genießt, nämlich das Internet of Things (zu deutsch: Internet der Dinge). Dieser Begriff beschreibt die fortschreitene Vernetzung von Produkten, Services, Maschinen und Sensoren mittels IP-basierter Protokolle. Ähnlich wie die Cloud spielt das IoT eine wichtige Rolle für die Industrie (smart factories, predictive maintenance) aber auch private Haushalte (smart Homes)\cite*[]{Next}\cite*[]{Öster}. 

Beide Themen sind für sich schon Trends die aktuell viel Aufmerksamkeit genießen. Umso interessanter ist die Kombination der beiden Themen. In einem IoT Netzwerk können riesige Mengen an Daten anfallen, welche nicht immer Zentral auf einem IoT Gerät verarbeitet werden können. Daher bietet sich das Verarbeiten der Daten in einer Cloud an. In der Cloud werden nun auf Grundlage der gesammelten Daten Berechnungen durchgeführt und beispielsweise mithilfe Künstlicher Intelligenz Aktionen ausgelöst. 

Thema dieser Ausarbeitung ist eben genau dieses Zusammenspiel der beiden Themenfelder. Hierzu wird zuerst auf die Grundlagen des Cloud Computings eingegangen und anschließend das Themenfeld IoT beleuchtet. Nach den Grundlagen wird kurz auf Amazon Web Services eingegangen und im Anschluss ein Beispielprojekt vorgestellt, in dem IoT Geräte mit AWS verknüpft werden.  
