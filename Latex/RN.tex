\documentclass[
	12pt,						 % Schriftgröße
	DIV=10,						 % Satzspiegelgröße Seitenränder
	german,						 % für Umlaute, Silbentrennung etc.
	paper=a4,					 % Papierformat
	oneside, 
	titlepage,					 % es wird eine Titelseite verwendet
	parskip=half,				 % Abstand zwischen Absätzen (halbe Zeile)
	headings=normal,			 % Größe der Überschriften verkleinern
	listof=totoc,				 % Verzeichnisse im Inhaltsverzeichnis aufführen
	bibliography=totoc,			 % Literaturverzeichnis im Inhaltsverzeichnis aufführen
	numbers=noenddot,			 % Kapitelnummern ohne Punkt	
	enabledeprecatedfontcommands,% Alte Befehle zur unterstüzung von \listofsymbols	
    notitlepage
]{scrreprt}

\makeatletter	
	\setlength\@fptop{0\p@}  % Entfernt Lücken am Seitenbeginn
\makeatother

\newcommand{\titel}{AWS-IoT}
\newcommand{\art}{Ausarbeitung Rechnernetze}
\newcommand{\autor}{Claas Meints, Magnus Müller}
\newcommand{\studienbereich}{Elektrotechnik/Mechatronik}
\newcommand{\comment}[1]{}
\newcommand{\TODO}{}
\newcommand\fig[2]{
	\begin{figure}[ht]
		\centering
		\includegraphics[width=0.8\textwidth]{figures/#1}
		\caption{#2} 
		\label{fig:#1}
	\end{figure}
}
% Zur Richtigen Verwendung von Einheiten
\usepackage[
    locale=DE, 
    %per=symbol-or-fraction
]{siunitx}

\sisetup{
    mode = text,
    detect-family,
    detect-weight,  
    exponent-product = \cdot,
    output-decimal-marker={\text{.}},
    %math-rm=\mathsf,
    %text-rm=\sffamily,
    range-phrase = { -- },
    per-mode=symbol,
    per-symbol=/
}
\DeclareSIUnit{\var}{var}

% Anpassung des Seitenlayouts -------------------------------------------------------------% 	siehe Seitenstil.tex
% -------------------------------------------------------------
\usepackage[headsepline,automark]{scrlayer-scrpage}

\usepackage{listings}
\usepackage{color}
% C++ color scheme
\definecolor{backgroundcolor}{RGB}{237, 241, 237}
\definecolor{commentcolor}{RGB}{0,128,0}
\definecolor{keywordcolor}{RGB}{136, 134, 58}
\definecolor{identifiercolor}{RGB}{24, 79, 110}
\definecolor{numbercolor}{RGB}{128,128,128}
\definecolor{preprocessorcolor}{RGB}{128,128,128}
\definecolor{stringcolor}{RGB}{163,21,21}

\lstdefinestyle{C++}{
    backgroundcolor=\color{backgroundcolor},   
    commentstyle=\color{commentcolor},
    directivestyle=\color{preprocessorcolor},
    keywordstyle=\color{keywordcolor},
    identifierstyle=\color{identifiercolor},
    numberstyle=\tiny\color{numbercolor},
    stringstyle=\color{stringcolor},
    basicstyle=\ttfamily\footnotesize,
    breakatwhitespace=false,         
    breaklines=true,                 
    captionpos=b,                    
    keepspaces=true,                 
    numbers=left,                    
    numbersep=5pt,                  
    showspaces=false,                
    showstringspaces=false,
    showtabs=false,                  
    tabsize=2
}
\lstset{language=C++,escapechar=@,style=C++}
\usepackage{booktabs}
\usepackage{eurosym}
\usepackage{pifont}% http://ctan.org/pkg/pifont

\renewcommand{\lstlistingname}{Code} % Damit Codebeispiele mit Code betitelt werden.


% Anpassung an Landessprache -------------------------------------------------------------% 	Verwendet globale Option german siehe \documentclass
% -------------------------------------------------------------
\usepackage[ngerman]{babel}
\usepackage{scrhack}
\usepackage[utf8]{inputenc}
\usepackage{amsmath,amssymb,units}
\usepackage{wrapfig,caption, placeins}
\usepackage{multicol}
\captionsetup{
    format = plain,
    justification = centering,
    labelsep = newline,
    singlelinecheck = false,
    labelfont = bf,
    font = small
}
\usepackage{mathptmx,charter,helvet,courier}
\usepackage[printonlyused, smaller]{acronym}
\usepackage{wrapfig}
\usepackage{pgfplots}
% Umlaute -------------------------------------------------------------% 		Umlaute/Sonderzeichen wie äöüß direkt im Quelltext verwenden (CodePage).
%		Erlaubt automatische Trennung von Worten mit Umlauten.
% -------------------------------------------------------------
\usepackage[utf8]{inputenc}
\usepackage[T1]{fontenc}
\usepackage{ae}			 	% "schöneres" ä
\usepackage{textcomp} 		% Euro-Zeichen etc.
\usepackage[german=quotes]{csquotes}


% Verwendung von Blindtext zur Layout gestaltung -------------------------------------------------------------
\usepackage{blindtext}

% Grafiken -------------------------------------------------------------
% 		Einbinden von Grafiken [draft oder final]
% 		Option [draft] bindet Bilder nicht ein - auch globale Option
% -------------------------------------------------------------
\usepackage{graphicx}
%\graphicspath{{Grafiken/}} 	% Dort liegen die Bilder des Dokuments


% Befehle aus AMSTeX für mathematische Symbole z.B. \boldsymbol \mathbb ----
\usepackage{amsmath,amsfonts}
\usepackage{tensor}

% Für Index-Ausgabe; \printindex -------------------------------------------------------------
\usepackage{makeidx}

% Einfache Definition der Zeilenabstände und Seitenränder -------------------------------------------------------------
\usepackage{setspace}
\usepackage{geometry}


%% Zum Umfließen von Bildern -------------------------------------------------------------
\usepackage{floatflt}


% Lange URLs umbrechen etc. -------------------------------------------------------------
\usepackage{url}

% für lange Tabellen
\usepackage{longtable}
\usepackage{multirow}
\usepackage{rotating}
\usepackage{diagbox}
\usepackage{array}
\newcolumntype{M}[1]{>{\centering\arraybackslash}m{#1}}
\usepackage{ragged2e}
\usepackage{lscape}
\usepackage{colortbl} %farbige hinterlegung


% Spaltendefinition rechtsbündig mit definierter Breite auf dem Deckblatt 
%------------------------------------------------------------
\newcolumntype{d}[1]{>{\raggedleft\hspace{0pt}}p{#1}}

% Formatierung von Listen ändern
\usepackage{paralist}
% Standardeinstellungen:
\setdefaultleftmargin{2.5em}{2.2em}{1.87em}{1.7em}{1em}{1em}

%Zur Verwendung von BibLaTex
\usepackage[
    style=alphabetic
]{biblatex}
\addbibresource{bibliografie.bib}
\DefineBibliographyStrings{ngerman}{
    andothers = {{et\,al\adddot}},             
}

\usepackage{xcolor}
\usepackage{caption}
\usepackage{float}
\usepackage{subfig}
\usepackage{xspace}
\usepackage{scrhack}

\usepackage[acronyms,toc]{glossaries} %Glossar Achtung Perl.exe muss installiert werden
\usepackage{glossaries-extra}
\usepackage{glossary-mcols}

% pdf-Optionen  -------------------------------------------------------------
\usepackage{pgfplots} % Für Kompatibilät unter verschiedenen OS
\pgfplotsset{compat=1.7}

\pdfminorversion=7 % Verwende pdf Version 1.7

\usepackage[
    bookmarks,
    bookmarksopen=true,
    bookmarksnumbered=true,
    colorlinks=true,
    %linkcolor=red, % einfache interne Verknüpfungen
    %anchorcolor=black,% Ankertext
    %citecolor=blue, % Verweise auf Literaturverzeichniseinträge im Text
    %filecolor=magenta, % Verknüpfungen, die lokale Dateien öffnen
    %menucolor=red, % Acrobat-Menüpunkte
    %urlcolor=cyan, 
    % für die Druckversion können die Farben ausgeschaltet werden:
    linkcolor=black, % einfache interne Verknüpfungen
    anchorcolor=black,% Ankertext
    citecolor=black, % Verweise auf Literaturverzeichniseinträge im Text
    filecolor=black, % Verknüpfungen, die lokale Dateien öffnen
    menucolor=black, % Acrobat-Menüpunkte
    urlcolor=black, 
    %backref,
    %pagebackref,
    plainpages=false,% zur korrekten Erstellung der Bookmarks
    pdfpagelabels,% zur korrekten Erstellung der Bookmarks
    hypertexnames=false,% zur korrekten Erstellung der Bookmarks
    linktoc=all%Sowohl Seitenzahl als auch Text als Link
]{hyperref}
\usepackage{bookmark}
\usepackage[all]{hypcap} % jump to top of figure
\usepackage{pdfpages}
% Zeilenabstand ------------------------------------------------------------
\onehalfspacing


% Seitenränder -------------------------------------------------------------
\geometry{paper=a4paper,left=25mm,right=25mm,top=30mm,bottom=40mm}

% Kopf- und Fußzeilen ------------------------------------------------------
\pagestyle{scrheadings}

% Kopf- und Fußzeile auch auf Kapitelanfangsseiten -------------------------
\renewcommand*{\chapterpagestyle}{scrheadings}

% Schriftform der Kopfzeile ------------------------------------------------
\renewcommand{\headfont}{\normalfont}

% Kopfzeile ----------------------------------------------------------------
\ihead{
	\parbox{11.8cm}{\normalsize{ \ }}\\
	%\small{\untertitel}\\[2ex]
	\textit{ \headmark}
}
\chead{
		%\includegraphics[height=0.8cm]{Grafiken/EWENETZ_Logo.png}
}
\ohead{	
		\includegraphics[height=1.4cm]{figures/Logo_PHWT.png}	
}
\setlength{\headheight}{21mm} % Höhe der Kopfzeile


% Fußzeile -----------------------------------------------------------------
\ifoot{
	%\includegraphics[height=1.5cm]{Grafiken/Logo_PHWT.png}
}
\cfoot{}
\ofoot{\pagemark}
\setlength{\footheight}{15mm}


% erzeugt ein wenig mehr Platz hinter einem Punkt --------------------------
%\frenchspacing 

% Schusterjungen und Hurenkinder vermeiden
\clubpenalty = 10000
\widowpenalty = 10000 
\displaywidowpenalty = 10000


%% Fußnoten fortlaufend durchnummerieren ------------------------------------
%\counterwithout{footnote}{chapter}
\setkomafont{section}{\large} 



\newglossary[llg]{glossary}{llo}{lls}{Glossar}
\newglossaryentry{Cloud Computing}
{
    name=Cloud Computing,
    description={},
    type=glossary
}
\newglossary[slg]{symbolslist}{syi}{syg}{Symbolverzeichnis}
\newglossaryentry{x}{
    name={\ensuremath{x}},
    description={Variable},
    type=symbolslist
}
\newglossary[wlg]{wordslist}{woi}{wog}{Fremdwortverzeichnis}

\makeglossaries
\newacronym{aws}{AWS}{Amazon Web Services}
\newacronym{iot}{IoT}{Internet of Things}


\begin{document}
\pagenumbering{Roman}
\thispagestyle{plain}
\begin{titlepage}
	\begin{center}
		\begin{minipage}{0.5\textwidth}
			\vspace{-1.5cm}
			\centering
			\includegraphics[width=7cm]{figures/LOGO_EWE.jpg}
		\end{minipage}%
		\hfill
		\begin{minipage}{0.5\textwidth}
			\vspace{-1.5cm}
			\centering
			\includegraphics[width=5.5cm]{figures/LOGO_PHWT.png}
		\end{minipage}
	
		\huge{\textbf{\art}}\\[1.5ex]
	
		\LARGE{\titel\footnote{Die Angabe zum Sperrvermerk erfolgt in der \hyperref[p2]{Erklärung zur Verwendung dieser Prüfungsarbeit/Abschlussarbeit}}}\\[2ex]
		%\LARGE{{\titeleng}}\\[2ex]

		\large{Private Hochschule für Wirtschaft und Technik Vechta/Diepholz\\
			in Kooperation mit \ewe}
		
		\vspace{0.8cm}
		\normalsize
		\onehalfspacing
		\begin{tabular}{d{5.4cm}p{6cm}}
			vorgelegt von:  & \quad \autor          \\
							& \quad Rudenbrook 2\\
							& \quad 26188 Edewecht\\
							& \quad claas.meints@ewe-netz.de\\
			Matrikelnummer: & \quad \matrikel       \\
			%Semester:       & \quad \semester       \\
			Studienbereich: & \quad \studienbereich \\
			Studiengang:    & \quad \studiengang    \\
			Erstgutachter:  & \quad \erstgutachter  \\
			%Zweitgutachter: & \quad \zweitgutachter \\
			Bearbeitungszeit:  & \quad 01.08.2021 bis 17.01.2022
		\end{tabular}
		
		\textcopyright\ 2021\\[1.5ex]
	\end{center}

	\singlespacing
	\small
	\noindent Dieses Werk einschließlich seiner Teile ist \textbf{urheberrechtlich geschützt}. Jede Verwertung außerhalb der engen Grenzen des Urheberrechtgesetzes ist ohne Zustimmung des Autors unzulässig und strafbar. Das gilt insbesondere für Vervielfältigungen, Übersetzungen, Mikroverfilmungen sowie die Einspeicherung und Verarbeitung in elektronischen Systemen. In dieses Werk darf ohne die ausdrückliche Zustimmung der \ewe\ und des Verfassers keine Einsicht durch Dritte, mit Ausnahme der Gutachter, genommen werden (\textbf{Arbeit mit Sperrvermerk}).
	
\end{titlepage}

\chapter{Einleitung}\label{ch:Einleitung}
    Mit Fortschreiten der Digitalisierung rücken datengenstützte, digitale Geschäftsmodelle immer weiter in den Fokus unternehmerischen Handelns. In den vergangenen Jahren sind in fast allen Wirtschaftsbereichen Daten als wichtige und wertvolle Ressource erkannt worden. Daten bilden häufig die Grundlage für neue Lösungsstrategien für bestehende Probleme und damit auch die Basis für neue Geschäftsmodelle.\\ Auch in der Energiewirtschaft gibt es ein Potenzial, viele Informationen aus Daten zu gewinnen und mit ihnen nicht nur energiewirtschaftliche Probleme anzugehen. Besonders interessante Ressourcen stellen die Daten der Energiezähler sowohl von gewerblichen Verbrauchern, aber auch von privaten Haushalten dar. So gibt es innerhalb des EWE Konzerns seit langem nicht nur den Wunsch, die Kosten für die Ablesung von Energiezählern durch Automatisierung zu senken, sondern auch verschiedene Ideen zur Entwicklung neuer Geschäftsmodelle auf Basis der Daten. Die Anwendungsmöglichkeiten hoch aufgelöster Verbrauchsdaten aller Sparten reichen von neuen Tarifmodellen bis hin zur Erkennung von An- bzw. Abwesenheit in einem Haushalt.\\ Das SiGI (Simple Generic Interpreter) Projekt, dem diese Arbeit angeschlossen ist, verfolgt das Ziel, hoch aufgelöste Verbrauchsdaten aller Sparten flächendeckend verfügbar zu machen. Konkret geht es um die Entwicklung eines Gerätes, welches die Messdaten analoger Zähler digital erfasst und diese über ein Backend bereitstellen kann. Der Fokus liegt dabei auf den Sparten Gas und Wasser, während in der Strom-Sparte bereits andere Lösungen etabliert werden.\\ Um eine generische Erfassung von Messdaten aller analogen Zählertypen zu gewährleisten, basiert das im Rahmen des SiGI Projektes entwickelte Gerät auf der Erfassung der von den Messgeräten angezeigten Zählerstände mittels KI-Bilderkennung. In einem Foto von einem Anzeigeinstrument des Zählers müssen dazu Ziffern erkannt werden. Diese Aufgabe wird durch ein Neuronales Netz (Modell) bewältigt.\\ Da für einen flächendeckenden Einsatz des Gerätes möglichst geringe Herstellungskosten erforderlich sind, sollte die im Gerät verwendete Hardware möglichst günstig sein. Ziel dieser Arbeit ist es, zu prüfen, ob die Entwicklung des Gerätes auf Basis eines Mikrocontrollers möglich ist. Ließe sich das Gerät auf Basis eines Mikrocontrollers realisieren, würde dies zu einer signifikanten Senkung der Kosten führen.\\ Ein besonderer Fokus in dieser Arbeit liegt auf der Prüfung, ob sich das im Rahmen von SiGI entwickelte Modell auf einem Mikrocontroller ausführen lässt. Die Detektion von Objekten in einem Bild, zu der auch die Detektion von Ziffern zählt, konnte bisher nicht auf einem Mikrocontroller realisiert werden.\\ Diese Arbeit ist in 6 Kapitel unterteilt. Im Kapitel~\nameref{ch:TGrundlagen und Stand der Technik} werden zunächst alle für die Arbeit relevanten Informationen gesammelt. Das Kapitel~\nameref{ch:Konzeption} beschreibt das Vorgehen bei der Problemlösung und die Konzepte, die in diesem Zuge entwickelt werden. Im Kapitel~\nameref{ch:Ergebnisse} werden die Ergebnisse der Arbeit betrachtet und bewertet und die erarbeiteten Konzepte verglichen. Das Kapitel~\nameref{ch:Ausblick} betrachtet die Folgen für das Projekt SiGI und die \nameref{ch:Zusammenfassung} fasst diese Arbeit zusammen. 
    \pagenumbering{arabic}
\chapter{Was ist \Gls{Cloud Computing}}\label{ch:2}
\chapter{\acrfull{iot}}\label{ch:3}
\section{Was ist \acrshort{iot}}\label{sec:3.1}
Unter \acrshort{iot} wird die Vernetzung vieler physicher Geräte mittels dem Internet verstanden. Der Begriff ist bereits seit 1999 in Gebrauch\cite*[]{IARJSET} und wird für eine vielzahl unterschiedlicher technischer Lösungen und Protokolle verwendet. Die verschiedenen technischen Lösungen eint dabei die Verwendung des \acrlongpl{ip} (\acrshort{ip}) innerhalb des \Gls{Protokoll-Stack}s\cite*[]{Rayes2019}.\\ Der erste Schritt in Richtung \acrshort{iot} ist dabei für gewöhnlich die Ausstattung von Geräten mit einer vielzahl von Sensoren. So können Geräte den Zustand ihrer Umgebung erfassen und diese über die Netzwerkverbindung an ein \Gls{Backend} übertragen. Dadurch wird es möglich physische Zustände lokal, aber potenziell auch von überall auf der Welt aus zu überwachen. Das Backend kann also zum einen einer Privatperson ermöglichen die eigenen Geräte zu überwachen, zum anderen können die Geräte auch als Datenquellen zum Beispiel für Unternehmen dienen, welche dann die Informationen vieler Geräte in einem \Gls{Cloud}-\Gls{Backend} bündeln\cite*[]{inproceedings}.\\ Für viele technische Anwendungen ist neben Sensordaten auch die Möglichkeit der Steuerung von Aktoren notwendig. Hierbei ist sowohl die Vernetzung mit anderen \acrshort{iot}-Geräten als auch die Vernetzung mit dem \Gls{Backend} wichtig. Über die Kopplung mit einer \acrshort{iot}-Sensoreinheit kann so zum Beispiel die Regelung des Aktors vorgenommen werden. Wird der Aktor an ein Backend angeschlossen, ist darüber eine automatische Fernsteuerung möglich\cite*[]{Rayes2022}.\\ Neben Sensoren und Aktoren sind Einheiten für die Verarbeitung von Daten notwendig. Dabei kann die Datenverarbeitung sowohl im \Gls{Backend} geschehen, als auch durch ein \acrshort{iot}-Gerät. Die Verarbeitungslogik kann sowohl in ein bestehendes \acrshort{iot}-Gerät integriert werden, als auch als separate Einheit ausgeführt sein. Durch die so geschaffene Möglichkeit der lokalen Verarbeitung von Daten wird das Gesamtsystem resilienter gegen den Ausfall von Netzwerkverbindungen. Auch die Latenz wird durch die direktere Verbindung im lokalen Netzwerk verringert\cite*[]{Rayes2022}.
\section{Anwendungsgebiete}\label{sec:3.2}
Durch die bereits beschriebene Vielfältigkeit der technischen Lösungen, die unter dem Begriff \acrshort{iot} zusammengefasst werden ergibt sich auch eine Vielzahl an Anwendungsgebieten und Einsatzmöglichkeiten.\\ Einer der präsentesten Bereiche in dem \acrshort{iot}-Geräte eingesetzt werden ist die Hausautomatisierung. Hier eingesetzte \acrshort{iot}-Geräte können zum einen dem Komfort der Benutzer*innen erhöhen, in dem Arbeiten im Haushalt automatisiert oder erleichtert werden. Darüber hinaus existieren bereits viele Lösungen, welche primär der Sicherheit gegen Einbruch etc. dienen. Eine weitere wichtige Aufgabe von \acrshort{iot}-Geräten im Haus ist die energetische Optimierung. So werden \glqq{}intelligente\grqq\ Heizsysteme mit Temperaturfühlern und Aktoren an den Rolläden gekoppelt, um die gewünschte Temperatur im Innenraum möglichst kostengünstig und Emissionsarm zu erreichen. Auch die Anpassung des elektrischen Energieverbrauchs an die Verfügbarkeit und den Preis von elektrischer Energie kann mittels \acrshort{iot} erfolgen\cite*[]{Zaheeruddin2020}.\\ Auch für die Sammlung medizinisch relevanter Daten werden vermehrt \acrshort{iot}-Geräte eingesetzt. Ein Beispiel kann die Messung von Blutruck oder die Durchfühung eines Elektrokardiogramms durch ein Armband sein. Da diese permanent getragen werden, können Gesundheitliche Probleme auch ohne Verdachtsfall erkannt werden. Eine rechtzeitige Behandlung wird so wahrscheinlicher\cite*[]{Zaheeruddin2020}.\\ In der Landwirtschaft können \acrshort{iot}-Geräte dazu verwendet werden zum Beispiel das Pflanzenwachstum zu überwachen. Durch die Erfassung der Daten kann optimal bewässert und gedüngt werden. Auch vollautomatische Systeme können mittels Steuerungstechnik realisiert werden. \acrshort{iot} bietet dabei die Möglichkeit Sensordaten zu erfassen und im \Gls{Cloud}-\Gls{Backend} mit Wetterdaten zu verknüpfen um langfristig die Bewässerung planen und Optimieren zu können\cite*[]{Zaheeruddin2020}.\\ Der Vermutlich wichtigste Einsatzbereich von \acrshort{iot} ist die Industrie. Der Einsatz von \acrshort{iot}-Geräten in der Industrie wird in Deutschland mit dem Begriff \glqq{}Industrie 4.0\grqq\ beschrieben. Mögliche aufgaben von \acrshort{iot}-Geräten sind die Bereitstellung von Daten zur Bestimmung wartungsintervallen oder aber auch die Automatisierung ganzer Produktionsstraßen. Viele Unternehmen streben dabei eine sogenannte \glqq{}Smart Factory\grqq\ an, also eine Produktionsstraße, welche unterschiedliche Produkte fertigen kann und die Produktionsdaten von einem oder mehreren in dem Produkt integrierten \acrshort{iot}-Geräten erhält. Alle an dem Fertigungsprozess beteiligten Geräte sind dabei untereinander und mit dem Produkt vernetzt. Im idealfall ist für ein Angepasstes Produkt nur eine Änderung der im Produkt gespeicherten Produktionsdaten notwendig, ohne dass die am Fertigungsprozess beteiligten Geräte neu Programmiert oder anderweitig angepasst werden müssen\cite*[]{Zaheeruddin2020}\cite*[]{10.1007/978-3-030-39875-0_20}.\\ % TODO: China?
\section{Technologische Umsetzung von \acrshort{iot}}\label{sec:3.3}
Da eine Ethernet-Verbindung für \acrshort{iot}-Geräte oft nicht verfügbar oder für den Anwendungsfall ungeeignet ist, kommunizieren ein Großteil der \acrshort{iot}-Geräte Drahtlos. So können auf den Layern Eins und Zwei zum Beispiel die Funkstandards IEEE 802.15.4 oder LPWAN bzw. LoRaWAN eingesetzt werden.
\chapter{Amazon Web Services (AWS)}\label{ch:4}

Das Unternehmen Amazon bietet selbst mehr als Erfolgreich Cloud Computing Dienste unter dem Namen Amazone Web Services an. Anfänglich wurden nicht genutzte Rechenkapazitäten verkauft, jedoch hat sich das Anbieten von IT-Ressourcen in der Cloud als extrem profitabel erwiesen und mittlerweile erzielt das Cloud Geschäft des Unternehmens mehr Gewinn als der bekannte Online Handel. Bei AWS handelt es sich aktuell um den Cloud Anbieter mit dem weltweit größten Marktanteil und derzeit werden den Benutzern mehr als 200 verschiedene Dienste angeboten. Somit stehen den Benutzern alle erdenklichen Services zur Verfügung\cite*[]{AMA1}.

Im Bereich IaaS bietet der Anbieter beispielsweise Services wie EC2 und S3 an. Mit EC2 wird dem Benutzer Rechenleistung in Form von Instanzen auf dem Server bereitgestellt und dieser hat dabei die vollständige Kontrolle über die virtuellen Ressourcen. Der Benutzer bezahlt hier lediglich die in Anspruch genommene Rechenleistung. Amazons simple Storage Service S3 bietet einen Objektspeicher mit Web-Schnittstellen. Es ist möglich große Datenmengen schnell in den Datenspeicher hinzuzufügen oder sie abzurufen. Neben S3 bietet Amazon noch weitere Speicherdienste wie z.B. Backup, welche weniger schnelle Zugriffe ermöglichen aber dadurch auch erheblich günstiger pro genutzte Speicherkapazität sind. \cite*[]{ÜAWS}

Auch oberhalb der genannten Ebenen bietet AWS Services wie Lambda an. Dieser ist oberhalb von SaaS in der Ebene Function as a Service einzuordnen, denn der Benutzer besitzt weniger Verantwortung als bei SaaS. Mithilfe von Lambda ist es möglich Code einfach Serverseitig durch definierte Trigger ausführen zu lassen. Der Kunde zahlt hier je nachdem wie oft der Code ausgeführt wird. \cite*[]{ÜAWS}

In Bezug auf das Thema IoT stellt Amazon eine Flotte an Services bereit. FreeRTOS und GreenGrass stellen hier Betriebssysteme dar, welche auf den IoT Geräten laufen und bereits vordefinierte Schnittstellen zur Kommunikation mit der Cloud besitzen. FreeRTOS dient hier als Echtzeitbetriebssystem für die einzelnen IoT Geräte und GreenGrass eher als ein Verwaltungsbetriebssystem z.B. zum Sammeln und Synchronisieren der erfassten Daten auf einem übergeordneten Gerät. \cite*[]{ÜAWS}

Um eine Steuerung und Überwachung in der Cloud zu ermöglichen dienen Dienste wie IoT Core, Fleetwise und Device Manegement. Durch diese lassen sich Flotten an IoT Geräten verwalten, Überwachen und auch Steuern. Zudem ermöglichen diese ein Sammeln der Daten und ermöglichen ein Updaten der Gerätesoftware von zentraler Stelle.\cite*[]{ÜAWS}

Zur weiteren Verarbeitung der Daten gibt es weitere Services wie IoT Sietwise, Analytics und Events. Beispielsweise lassen sich Daten für das weitere Verarbeiten mithilfe von Machine learning tools vorbereiten durch Filtern, Transformieren und Anreichern. Aus den gesammelten Daten lassen sich Aktionen erstellen, die dann wieder von den IoT Geräten ausgeführt werden. Ein Anwendungsfall hierfür ist eine Maschine in einer Produktionshalle. Durch das Auswerten der gesammelten Daten der gesamten Sensorik, lassen sich Rückschlüsse auf den Zustand der Maschine schließen. Somit ist es möglich vollständig automatisiert in die Produktion einzugreifen, sobald eine Unregelmäßigkeit im Betrieb festgestellt wird.\cite*[]{ÜAWS}
\input{chapters/4 ESP32 + AWS Core IoT.tex}
\input{chapters/5 Zusammenfassung.tex}

\listoffigures
% \listoftables
\printglossary[type=\acronymtype,title={Abkürzungsverzeichnis}] % list of acronyms
% \printglossary[type=symbolslist, style=long] % list of symbols
% \printglossary[type=wordslist,   style=long] % list of english words
\printglossary[type=glossary,style=mcoltreenoname] % main glossary
\printbibliography

\appendix
\chapter{Anhang}\label{ch:A}
\section{Quizz Buzzer Codebeispiel (gekürzt)}\label{sec:code}
\lstinputlisting[language=C++]{code/subscribe_publish_sample.c}\newpage

\end{document}